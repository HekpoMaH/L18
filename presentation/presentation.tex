\documentclass{beamer}
\usepackage{subcaption}
\usepackage{natbib}
\usepackage{tikz}
\usepackage{tikz-qtree}
\usepackage{mathtools}
\usepackage{textpos}
\usepackage{stackengine} 
\newcommand\oast{\stackMath\mathbin{\stackinset{c}{0ex}{c}{0ex}{\ast}{\bigcirc}}}
%\usepackage{pifont}
\newcommand{\smallspace}{\hspace{1mm}}
\setbeamercolor{alerted text}{fg=blue}

\graphicspath{{./images/}}
\mode<presentation> {
\usetheme{Madrid}

% \usecolortheme{albatross}
% \usecolortheme{beaver}
% \usecolortheme{beetle}
% \usecolortheme{crane}
% \usecolortheme{dolphin}
% \usecolortheme{dove}
% \usecolortheme{fly}
% \usecolortheme{lily}
% \usecolortheme{orchid}
% \usecolortheme{rose}
% \usecolortheme{seagull}
% \usecolortheme{seahorse}
% \usecolortheme{whale}
% \usecolortheme{wolverine}

% \setbeamertemplate{footline} % To remove the footer line in all slides uncomment this line
% \setbeamertemplate{footline}[page number] % To replace the footer line in all slides with a simple slide count uncomment this line

\setbeamertemplate{navigation symbols}{} % To remove the navigation symbols from the bottom of all slides uncomment this line
}

%\usepackage{graphicx} % Allows including images
\usepackage{booktabs} % Allows the use of \toprule, \midrule and \bottomrule in tables

%%----------------------------------------------------------------------------------------
%%	TITLE PAGE
%%----------------------------------------------------------------------------------------

\title[RL of Theorem Proving]{Reinforcement Learning of Theorem Proving} % The short title appears at the bottom of every slide, the full title is only on the title page

%\author{Dobrik Georgiev} % Your name
%\institute[Cambridge] % Your institution as it will appear on the bottom of every slide, may be shorthand to save space
%{
%\textit{dgg30@cam.ac.uk.com} % Your email address
%}
%\date{} % Date, can be changed to a custom date

\begin{document}

\begin{frame}
\titlepage % Print the title page as the first slide
\end{frame}

\begin{frame}
\frametitle{Overview} % Table of contents slide, comment this block out to remove it
\tableofcontents % Throughout your presentation, if you choose to use \section{} and \subsection{} commands, these will automatically be printed on this slide as an overview of your presentation
\end{frame}
\section{How tableau provers work} 

\begin{frame}
    \frametitle{How tableau provers work}
    \begin{figure}
    \includegraphics[width=\linewidth]{tableaux.png}
    \caption{The search tree of a tableau based ATP\footnote{Source: Wikipedia}}
    \end{figure}

\end{frame}
% \begin{frame}
%     \frametitle{Connection Tableau - how it works}
%     Assume the classical proof by contradiction setting
%     \begin{itemize}
%         \item Translate the formula to CNF
%         \item Skolemize
%         \item Represent as set of clauses
%         \item Proof proceeds by building tableau with \alert{connections}
%             \begin{itemize}
%                 \item ($L$, $\neg{L}$) is a \alert{connection}
%                 \item We will see an example
%             \end{itemize}
%         \item Successful proof $\iff$ connection in every path from root to leaf
%     \end{itemize}
% \end{frame}

% \begin{frame}
%     \frametitle{Connection Tableau - an example}
%     \begin{columns}
%         \column{.45\textwidth}
%         Consider\footnotemark:
%         \begin{align*}
%             &\begin{aligned}
%                 \{P,\smallspace R\} \smallspace
%                 \{\neg{P},\smallspace X \}\smallspace
%                 \{\neg{b},\smallspace P\}\smallspace
%             \end{aligned}\\
%             &\begin{aligned}
%                 \{\neg{c },\smallspace \neg{P}\}\smallspace
%                 \{P,\smallspace \neg{R}\}
%             \end{aligned}
%         \end{align*}

%         \begin{itemize}
%             \item<2-> Start rule
%             \item<3-> Extension rule
%             \item<5-> Reduction rule
%         \end{itemize}
%         \column{.45\textwidth}
%                 \centering
%                 \only<2>{
%                     \begin{tikzpicture}[remember picture, overlay]
%                     \tikzstyle{level 1} = [level distance=.8cm, sibling distance=30mm]
%                     \tikzstyle{level 2} = [level distance=1cm, sibling distance=10mm]
%                     \tikzstyle{level 3} = [level distance=1cm, sibling distance=10mm]
%                     % \tikzset{every tree node/.style={anchor=center, scale=0.84}}
%                     \node (s) at (0, 2) {}
%                         child { node{$P$} }
%                         child { node{$R$} };
%                     \end{tikzpicture}%
%                 }%
%                 \only<3>{
%                     \begin{tikzpicture}[remember picture, overlay]
%                     \tikzstyle{level 1} = [level distance=.8cm, sibling distance=30mm]
%                     \tikzstyle{level 2} = [level distance=1cm, sibling distance=10mm]
%                     \tikzstyle{level 3} = [level distance=1cm, sibling distance=10mm]
%                     % \tikzset{every tree node/.style={anchor=center, scale=0.84}}
%                     \node (s) at (0, 2) {}
%                         child { node{$P$}
%                             child { node{$\neg{P}$} }
%                             child { node{$X'$} } }
%                         child { node{$R$} };
%                     \end{tikzpicture}%
%                 }%
%                 \only<4-5>{
%                     \begin{tikzpicture}[remember picture, overlay]
%                     \tikzstyle{level 1} = [level distance=.8cm, sibling distance=30mm]
%                     \tikzstyle{level 2} = [level distance=1cm, sibling distance=10mm]
%                     \tikzstyle{level 3} = [level distance=1cm, sibling distance=10mm]
%                     % \tikzset{every tree node/.style={anchor=center, scale=0.84}}
%                     \node (s) at (0, 2) {}
%                         child { node (p) {$P$}
%                             child { node {$\neg{P}$} }
%                             child { node{$X'$}
%                                 child{ node{$\neg{c}$} }
%                             child{ node (negp) {$\neg{P}$} } } }
%                         child { node{$R$} };
%                     \only<5>{\draw[densely dashed] (p) to [bend left=50,auto] (negp);}
%                     \end{tikzpicture}%
%                 }%
%                 \only<6>{
%                     \begin{tikzpicture}[remember picture, overlay]
%                     \tikzstyle{level 1} = [level distance=.8cm, sibling distance=30mm]
%                     \tikzstyle{level 2} = [level distance=1cm, sibling distance=10mm]
%                     \tikzstyle{level 3} = [level distance=1cm, sibling distance=10mm]
%                     % \tikzset{every tree node/.style={anchor=center, scale=0.84}}
%                     \node (s) at (0, 2) {}
%                         child { node{$P$}
%                             child { node{$\neg{P}$} }
%                             child { node{$X'$}
%                                 child{ node{$\neg{c}$} }
%                                 child{ node{$\neg{P}$} } } }
%                         child { node{$R$}
%                             child { node{$P$}
%                                 child { node{$\neg{P}$} }
%                                 child { node{$X''$}
%                                     child{ node{$\neg{c}$ } }
%                                     child{ node{$\neg{P}$ } }
%                                 }
%                             }
%                             child { node{$\neg{R}$} } };
%                     \end{tikzpicture}%
%                 }%
%     \end{columns}
%     \footnotetext[1]{Example adapted from\\ \url{http://www.leancop.de/atp-fub12/atp_fub12_3cop.pdf}}
% \end{frame}
% \subsection{Semantics of Connection Tableau}

% \subsection{Searching for a Closed Tableau}

\section{Reinforcement Learning}

\begin{frame}
    \frametitle{Basics}
    \begin{figure}
    \includegraphics[width=.7\linewidth]{RL_robot.png}
    \caption{An agent has to reach a reward without burning\footnote{Source: GeeksForGeeks}}
    \end{figure}
\end{frame}
\subsection{Basics}

\subsection{Application to ATP}
\begin{frame}
    \frametitle{Application to ATP}
    RL to ATP mapping:
    \begin{itemize}
        \item agent $\leftrightarrow$ ATP
        \item environment $\leftrightarrow$ search tree
        \item actions $\leftrightarrow$ extending search tree
        \item reward $\leftrightarrow$ finding a closed tableau
    \end{itemize}
\end{frame}

\begin{frame}
    \frametitle{Application to ATP -- the UCT formula}
    For each node $i$, we calculate the Upper Confidence bound applied to
    Trees, i.e. UCT\citep{UCT}:
    \begin{columns}
    \column{.6\textwidth}
    \begin{equation*}
        \frac{w_i}{n_i}+c\cdot p_i \cdot \sqrt{\frac{\ln N_i}{n_i}}
    \end{equation*}
    \column{.39\textwidth}
    \begin{itemize}
        \item \alert{$w_i$ -- total reward}
        \item $n_i$ -- number of visits
        \item $c$ -- hyperparameter
        \item \alert{$p_i$ -- prior probability}
        \item $N_i$ -- total parent visits
    \end{itemize}
    \end{columns}
    % Without any learning, all prior probabilities are equal and the values are
    % determined by the number of open goals.
\end{frame}

\begin{frame}
    \frametitle{Application to ATP -- Learning the parameters}
    \framesubtitle{\textbf{Step 1:} Extracting features}
    \begin{columns}
    \column{.7\textwidth}
    For each Literal $L$, e.g. $f(X, Y) = g(sk_1, sk_2(x))$ \footnotemark
    \begin{itemize}
        \item Build it's \alert{feature tree}
        \item Count the \alert{term walks} of length 3
            \begin{itemize}
                \item E.g. $(=, f, \oast)$ occurs twice
            \end{itemize}
        \item Count the occurrences of each triple, create literal and clause feature vectors
        \item Feature vector for each state contains:
            \begin{itemize}
                \item the triplet occurrence counts for each of its clauses and goals
                \item additional metadata, e.g. number of goals, most common symbols, etc.
            \end{itemize}
    \end{itemize}
    \column{.2\textwidth}
    \begin{figure}
    \includegraphics[width=\textwidth]{AST.png}
    \caption{Feature Tree}
    \end{figure}
    \end{columns}
    \footnotetext[3]{Example from \cite{Enigma}}

    % \textbf{Step 2:} Getting the data:
    %     \begin{itemize}
    %         \item Run many Monte-Carlo proof searches
    %         \item 
    %     \end{itemize}
\end{frame}
\begin{frame}
    \frametitle{Application to ATP -- Learning the parameters}
    \framesubtitle{\textbf{Steps 2\&3:} Data Extraction and Lerning}
    \begin{itemize}
        \item Unrestricted ATP runs to accumulate (some) UCT proof data
        \item Associate node state and action features 
            with action `relevance'
            \begin{itemize}
                \item $\dfrac{total\ frequency\ of\ action}{action\ frequency\ at\ node}$
            \end{itemize}
        \item Associate node state feature with value
            \begin{itemize}
                \item 0 if node not a proof
                \item $0.99^{proof\ depth}$ otherwise
            \end{itemize}
        \item Apply regression on the logits to learn `relevance' and value
    \end{itemize}
\end{frame}

\subsection{Results}

\begin{frame}
\frametitle{Results}
Results from 2003 problems of the Mizar Mathematical Library \citep{MML} with
limit of $2\times10^6$ inferences.
\begin{table}
\begin{tabular}{l l l l l l}
\toprule
Iteration & 1 & 5 & 10 & 15 & 20 \\
Proved & 1037 & 1182 & 1210 & 1223 & 1235 \\
\bottomrule
\end{tabular}
\caption{Proved problems per iterations of learning}
\end{table}

\begin{table}
\begin{tabular}{l l l}
\toprule
\textbf{Methodology} & \textbf{Proved} & \textbf{IPS} \\
\midrule
Heuristics & 876 & 64K \\
RL & 1235 & 16K \\
\bottomrule
\end{tabular}
\caption{Using RL gives 40\% more proves but slows down the inference speed}
\end{table}
\end{frame}

\section{Summary}

\begin{frame}
\frametitle{Summary}
\begin{itemize}
    \item Reinforcement Learning can be applied to tableau based provers
    \item Many new problems solved
    \item RL (and ML methods in general) slow down provers
\end{itemize}

\end{frame}
\begin{frame}
\Huge
\centering
Questions?
\end{frame}
%% TODO DELETE BELOW
%\begin{frame}
%\frametitle{Paragraphs of Text}
%\end{frame}

%%------------------------------------------------


%\begin{frame}
%\frametitle{Blocks of Highlighted Text}
%\begin{block}{Block 1}
%Lorem ipsum dolor sit amet, consectetur adipiscing elit. Integer lectus nisl, ultricies in feugiat rutrum, porttitor sit amet augue. Aliquam ut tortor mauris. Sed volutpat ante purus, quis accumsan dolor.
%\end{block}

%\begin{block}{Block 2}
%Pellentesque sed tellus purus. Class aptent taciti sociosqu ad litora torquent per conubia nostra, per inceptos himenaeos. Vestibulum quis magna at risus dictum tempor eu vitae velit.
%\end{block}

%\begin{block}{Block 3}
%Suspendisse tincidunt sagittis gravida. Curabitur condimentum, enim sed venenatis rutrum, ipsum neque consectetur orci, sed blandit justo nisi ac lacus.
%\end{block}
%\end{frame}

%%------------------------------------------------

%\begin{frame}
%\frametitle{Multiple Columns}
%\begin{columns}[c] % The "c" option specifies centered vertical alignment while the "t" option is used for top vertical alignment

%\column{.45\textwidth} % Left column and width
%\textbf{Heading}
%\begin{enumerate}
%\item Statement
%\item Explanation
%\item Example
%\end{enumerate}

%\column{.5\textwidth} % Right column and width
%Lorem ipsum dolor sit amet, consectetur adipiscing elit. Integer lectus nisl, ultricies in feugiat rutrum, porttitor sit amet augue. Aliquam ut tortor mauris. Sed volutpat ante purus, quis accumsan dolor.

%\end{columns}
%\end{frame}

%%------------------------------------------------
%\section{Second Section}
%%------------------------------------------------

%\begin{frame}
%\frametitle{Table}

%\begin{table}
%\begin{tabular}{l l l}
%\toprule
%\textbf{Treatments} & \textbf{Response 1} & \textbf{Response 2}\\
%\midrule
%Treatment 1 & 0.0003262 & 0.562 \\
%Treatment 2 & 0.0015681 & 0.910 \\
%Treatment 3 & 0.0009271 & 0.296 \\
%\bottomrule
%\end{tabular}
%\caption{Table caption}
%\end{table}

%\end{frame}

%%------------------------------------------------

%\begin{frame}
%\frametitle{Theorem}
%\begin{theorem}[Mass--energy equivalence]
%$E = mc^2$
%\end{theorem}
%\end{frame}

%%------------------------------------------------

%\begin{frame}[fragile] % Need to use the fragile option when verbatim is used in the slide
%\frametitle{Verbatim}
%\begin{example}[Theorem Slide Code]
%\end{example}
%\end{frame}

%%------------------------------------------------

%\begin{frame}
%\frametitle{Figure}
%Uncomment the code on this slide to include your own image from the same directory as the template .TeX file.
%%\begin{figure}
%%\includegraphics[width=0.8\linewidth]{test}
%%\end{figure}
%\end{frame}

%%------------------------------------------------

%\begin{frame}[fragile] % Need to use the fragile option when verbatim is used in the slide
%\frametitle{Citation}
%\cite{MaLARea}
%% An example of the \verb|\cite| command to cite within the presentation:\\~

%% This statement requires citation \cite{p1}.
%\end{frame}

%%------------------------------------------------

%%------------------------------------------------

\begin{frame}{Bibliography}
\bibliographystyle{apalike}
\bibliography{../refs.bib}
\end{frame}

%----------------------------------------------------------------------------------------

\end{document} 
